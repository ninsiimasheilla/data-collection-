

\documentclass[12pt,]{article}
\usepackage{zed-csp,graphicx,color}%from
\pagenumbering{roman}
\begin{document}

\begin{titlepage}
\centerline{Establishment of an online supplier payment system in supermarkets of Uganda to do away with cash payment effects.\\}
\paragraph*{•}
\centerline{  Prepared by:  Ninsiima sheilla 16/U/964 216001139.\\}
\paragraph*{•}
\paragraph*{•}
  \begin{flushright}
  The Report,\\
  DATE: $February,23^{rd},2018$.
 \tableofcontents

  \end{flushright}
\date{\today}
\end{titlepage}

\newpage





\pagenumbering{arabic}
\section{Introduction}
A supermarket is a self-service shop offering a wide variety of food and household products, organized into aisles.
Suppliers normally supply the supermarket with their products and are given money in return where they usually take the money by cash. Those suppliers that provide on credit later comeback to pick their money on a particular date communicated by the manager.


\section{Statement of the problem}
suppliers incur more costs while travelling back to the supermarket to pick payment and also their money gets stolen at times on their way back home.cash payment method hasn't favoured the suppliers because at times they appear for payment and don't find the manager responsible.
\section{Main objectives of the study}
To find out how the hardships suppliers face with the cash payment method.
\section{specific objective of the study}

To find out the reasons as to why cash payment method is used.

To examine the reasons as to why some suppliers give their goods on credit.


\section{Significance of the study}
this study is important because it aims at improving the method of payment for suppliers thus doing away with the negative effects faced with cash payment method.
\section{Scope of the study}
This study was mainly based on the supplier payment method in most supermarkets in Uganda where I did my research on Gods plan supermarket in Makindye, Mega standard supermarket opposite the old taxi park in Kampala town.
\section{literature view}
Gods plan supermarket purchases most of its goods from suppliers on credit and it issues payment for these goods on Tuesday every week. The suppliers always come to pick the money in person which money is given to them in cash form.
I found out that the cash payment method has some negative effects namely;
Loss of money,as the suppliers get back to their homes they tend to lose the money as some of it is always stolen.
Some suppliers come back for second payment and end up stealing the supermarket money claiming they hadn't picked money.
\section{Methodology}
The data will be obtained using the ODK collect and it will later be uploaded on the ODK aggregate server.
Different kinds of data will be collected including the companys' name,address,account number,suppliers'image,amount of money needed
\section{Recommendations }
Based on the findings and conclusions in this study, the following recommendations are made:

\begin{enumerate}

\item Supermarkets should adopt an online supplier payment system so that they can just send the money direct to the supplier’s account so as to reduce on congestion on the day of payment.
\item Supermarkets should also adopt the cheque payment method mostly for those getting large sums of money thus reducing on theft of the supplier money.


\end{enumerate}

\end{document}






